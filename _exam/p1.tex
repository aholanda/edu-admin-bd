\input faculdade

Baseando-se na Figura~\ref{fig:faculdade}, levando em conta estas
  restrições e supondo que o banco de dados já possui algumas
  entradas, escreva os comandos SQL que realize as seguintes operações:

  \question{1,5} Criar as tabelas (relações) da
  Figura~\ref{fig:faculdade}.

  \question{0,25} Selecionar todos os atributos de todos os alunos.

  \question{0,25} Selecionar o código e nome de todas as as
  disciplinas com carga horária entre 60 e 80 horas.

  \question{0,25} Selecionar o nome e email de todos os alunos que
  possua o domínio 'faculdade.edu.br' no endereço.

  \question{0,25} Selecionar o nome e RA de todos os alunos cujo
  nome não comece com a letra 'T'.
  
  \question{0,25} Apagar a tupla do aluno com RA igual a {\tt
    1234567}.

  \question{0,25} Atualizar o email do aluno com RA igual a {\tt
    7654321}, para {\tt vidigal@faculdade.edu.br}.
  
  \question{0,25} Selecionar o RA e nome de todos os alunos que já
  cursaram 'Matemática Discreta'.
  
  \question{0,25} Selecionar o nome e email de todos os professores que
  já lecionaram 'Redes de Computadores'.

  \question{0,5} Selecionar o nome e email de todos os alunos que
  ingressaram depois de 2013 e de todos os professores que foram
  admitidos depois de 2005;

  \question{0,5} Calcular a média das notas dos alunos para o 1$^o$ 
  bimestre de 2016 para a disciplina de 'Processamento de Imagens'.

  \question{0,75} Contar o número de alunos reprovados na disciplina
  'Estatística' no ano de 2013.

  \question{1,0} Contar o número de alunos aprovados na disciplina
  'Cálculo I' no 1$^o$ bimestre do ano de 2014.

  \question{0,75} Listar o nome das disciplinas e dos alunos que
  obtiveram nota maior do que 9,5 entre 2013 e 2016.
  
  \question{1,5} Selecionar os nomes e notas dos alunos  do 1$^o$ bimestre
  e do 1$^o$ semestre de 2016, ordenando por nota.
  
  \question{1,5} Selecionar todos os alunos que já tiveram ou têm aulas 
  com o professor 'Ulisses Ptolomeu';
  

