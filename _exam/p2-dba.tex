%\examheader{Administração de Banco de Dados}{EXAME}{15/12/2015}

\question{5}~Crie uma tabela usando a declaração SQL a seguir:

\begin{verbatim}
CREATE TABLE produto (id SERIAL, categoria CHAR, preco REAL );
\end{verbatim}

\noindent e insira os valores

\begin{verbatim}
INSERT INTO produto(categoria,preco) VALUES('A', 120);
INSERT INTO produto(categoria,preco) VALUES('B', 240);
INSERT INTO produto(categoria,preco) VALUES('C', 360);
INSERT INTO produto(categoria,preco) VALUES('D', 440);
INSERT INTO produto(categoria,preco) VALUES('A', 80);
INSERT INTO produto(categoria,preco) VALUES('C', 200);
INSERT INTO produto(categoria,preco) VALUES('B', 300);
\end{verbatim}

\noindent Faça um função que atualize o preço do produto de acordo
com a categoria.

\bigskip
\noindent Para as tabelas

\begin{verbatim}
CREATE TABLE functionario(id SERIAL PRIMARY KEY, nome VARCHAR(128) NOT NULL);
CREATE TABLE auditoria(
       id SERIAL PRIMARY KEY, 
       produto_id INT REFERENCES produto(id) NOT NULL, 
       funcionario_id INT REFERENCES funcionario(id) NOT NULL,
       data_alteracao TIMESTAMP NOT NULL
);
\end{verbatim}

\noindent faça um função que armazene na tabela {\tt auditoria}
todas alterações realizadas na tabela {\tt produto}, com o {\tt id}
do funcionário que realizou a alteração e data.


\end{document}

\question{5} Escreva as declarações SQL que executem as tarefas a seguir:

\begin{enumerate}[a)]
\item Criar um usuário chamado {\tt chico} que possa se conectar, 
criar banco de dados e que transmita atributos (herança).
\item Crie um banco de dados chamado {\tt loja} e conceda permissões 
de administração para o usuário {\tt chico}.
\item Crie um grupo chamado {\tt vendas} que não transmita atributos (herança) 
e adicione o usuário {\tt chico} nele.
\item Crie um grupo chamado {\tt marketing} que não transmita atributos (herança) e adicione o grupo vendas nele.
\item Crie um usuario {\tt leitor} com permissão de conexão e que transmita atributos. Crie a tabela {\tt venda} do exercício anterior no banco de dados {\tt loja} e conceda somente privilégios de leitura para o usuário {\tt leitor}.
\end{enumerate} 

\question{3} Desejamos armazenar o número de visitas às diferentes
páginas web de uma empresa. Para isso, criamos a tabela a seguir:

\begin{verbatim}
CREATE TABLE paginaweb (
       url VARCHAR(256) PRIMARY KEY,
       visitas INT NOT NULL
);
\end{verbatim}

O problema ao armazenar as visitas é que o servidor web normalmente é
multi-thread, ou seja, atende a várias requisições simultaneamente. Assim,
a conconrrência de acesso deve ser levada em conta durante a atualização
do número de visitas.

Para as declarações SQL a seguir:

\begin{verbatim}
INSERT INTO paginaweb VALUES('prova.com', 333); -- valor inicial
\end{verbatim}

\begin{center}
\begin{tabular}{l|l}\hline
\bf \hfil Cliente 1 &\bf Cliente 2 \\\hline
 & \tt BEGIN \$mode;\\
 \tt BEGIN \$mode; & \\
\tt UPDATE paginaweb SET visitas = visitas + 1 &\\
\tt WHERE url = 'prova.com' & \\
 & \\
 & \tt UPDATE paginaweb SET visitas = visitas + 1 \\
 & \tt WHERE url = 'prova.com'  \\
\tt COMMIT; &\\
 &\tt COMMIT \\\hline
\end{tabular}
\end{center}

O que ocorrerá com o valor do campo {\tt visitas} se a variável
{\tt \$mode} receber os seguintes valores:

\begin{enumerate}[a)]
\item {\tt ''};
\item {\tt 'TRANSACTION ISOLATION LEVEL REPEATABLE READ'};
\item {\tt 'TRANSACTION ISOLATION LEVEL SERIALIZABLE'}?
\end{enumerate}

\question{2} Usando o exemplo do exercício anterior, a aplicação
no cliente pode executar as declarações da seguinte forma: 

\begin{center}
\begin{tabular}{l|l}\hline
\bf \hfil Cliente 1 &\bf Cliente 2 \\\hline
 & \tt BEGIN \$mode; \\
 \tt BEGIN \$mode; & \\
\tt SELECT visitas FROM paginaweb & \\
\tt WHERE url = 'prova.com'; & \\
\tt -- cacula visitas internamente & \\
\tt -- \$novo\_valor = \$visitas + 1; & \\
\tt UPDATE paginaweb SET visitas = \$novo\_valor &\\
\tt WHERE url = 'prova.com' & \\
 & \\
 & \\
&\tt SELECT visitas FROM paginaweb  \\
&\tt WHERE url = 'prova.com';  \\
&\tt -- cacula visitas internamente  \\
&\tt -- \$novo\_valor = \$visitas + 1;  \\
 & \tt UPDATE paginaweb SET visitas = \$novo\_valor \\
 & \tt WHERE url = 'prova.com'  \\
\tt COMMIT; &\\
 &\tt COMMIT \\\hline
\end{tabular}
\end{center}

O que ocorrerá com o valor do campo {\tt visitas} se a variável
{\tt \$mode} receber os seguintes valores:

\begin{enumerate}[a)]
\item {\tt ''};
\item {\tt 'TRANSACTION ISOLATION LEVEL REPEATABLE READ'};
\item {\tt 'TRANSACTION ISOLATION LEVEL SERIALIZABLE'}?
\end{enumerate}

\question{2} Se no exercício anterior for utilizado o comando
{\tt SELECT ... FOR UPDATE} ao invés de apenas {\tt SELECT}, qual
será o efeito sobre o valor de {\tt visita} para os níveis de isolamento
dos itens a) e b)?

\vfil Para pensar: Usando o nível de isolamento {\tt TRANSACTION ISOLATION LEVEL SERIALIZABLE}, 
é provocado erro se houver tentativa de atualização de 2 transações
concorrentes.  Pense em como este erro poderia ser tratado, para que a
atualização na transação em que o erro foi lançado possa ocorrer após
o término da transação concorrente. Esta solução para o acesso
concorrente é chamada de bloqueio otimista, enquanto que o bloqueio
realizado usando {\tt SELECT ... FOR~UPDATE} é chamado de pessimista.

\end{document}

\question{4} Um banco de dados possui uma tabela chamada {\tt aluno}
que possui um campo {\tt id} do tipo inteiro de 4 bytes, além de
outros campos. Supondo que neste banco de dados o campo {\tt id} é
chave primaria e automaticamente é indexada por uma árvore B de
ordem~{\bf 5}.  Responda às seguintes questões:

\begin{enumerate}[a)]
\item {\tt [3]} Construa a árvore gerada após a inserção dos
  seguintes valores: 70, 83, 81, 75, 67, 76, 72, 84, 86, 87, 77, 82,
  80, 65, 66, 88, 89, 68, 90, 69, 8, 49, 34, 23, 20, 28, 50, 11 e 102.

\item {\tt [1]} Calcule o espaço ocupado pelo índice se o ponteiro
  para as subárvores ocupar 4 bytes.
\end{enumerate}

\question{2} Para o exercício anterior, se a tabela {\tt aluno}, além
de {\tt id}, conter os campos {\tt nome} (32 bytes), {\tt endereço}
(40 bytes), {\tt sexo} (1 byte), {\tt departamento} (16 bytes), {\tt
  CPF} (9 bytes), {\tt data\_de\_nascimento} (8 bytes) e {\tt
  data\_de\_admissão} (8 bytes), quantas páginas de 8.192 bytes serão
ocupadas se a tabela possuir 40.000 registros?


\question{1} Quais são os 3 métodos de varreduras no PostgreSQL?

\begin{enumerate}[a)]
\item sequencial, indexada com índices primários e mapa de bits, indexada com índices secundários
\item indexada com índices primários, sequencial, indexada com índices secundários
\item sequencial, indexada com árvore B, indexada com árvore B e mapa de bits
\item ordenada, indexada com árvore binária, indexada com índices secundários
\item ordenada, indexada com árvore B, indexada com índices primários
\end{enumerate}

\question{1} A seguir são listadas 2 séries de operações concorrentes
em um banco de dados PostgreSQL sobre a tabela {\tt bar} que
possui somente o campo {\tt id} do tipo inteiro. Descrever o que
ocorre na série de operações da direita que provoca sua espera.

\begin{minipage}[h]{.5\linewidth}
\begin{verbatim}
postgres=# BEGIN;

postgres=# UPDATE bar SET id=id+1;
UPDATE 1




postgres=# COMMIT;
COMMIT
\end{verbatim}
\end{minipage}
\begin{minipage}[h]{.5\linewidth}\vspace{1.5cm}
\begin{verbatim}
postgres=# BEGIN;




postgres=# UPDATE bar SET id=id+1;
-- espera
-- espera
-- espera
UPDATE 1
postgres=# COMMIT;
\end{verbatim}
\end{minipage}


\pagebreak \question{2} {\tt [1]} As 2 séries de operações
concorrentes em um banco de dados PostgreSQL listadas a seguir ocorrem
sobre a tabela {\tt custo} que possui os campos {\tt id} e {\tt
  valor}, ambos do tipo inteiro. Explicar o que ocorre na série de
operações da direita que provoca o erro. {\tt [1]} Como este erro
poderia ser evitado?

\bigskip\footnotesize
\begin{minipage}[h]{.5\linewidth}
\begin{verbatim}
-- transação A
postgres=# BEGIN;
BEGIN
postgres=# UPDATE custo SET valor=valor-100 
           WHERE id = 1;
UPDATE 1
postgres=# UPDATE custo SET valor=valor+100 
           WHERE id = 2;
-- espera
-- espera
-- espera
-- espera
-- espera

UPDATE 1
\end{verbatim}
\end{minipage}
\begin{minipage}[h]{.5\linewidth}
\begin{verbatim}
-- transação B
postgres=# BEGIN;
BEGIN
postgres=# UPDATE custo SET valor=valor-150 
           WHERE id = 2;



UPDATE 1
postgres=# UPDATE custo SET valor=valor+150 
           WHERE id = 1;
ERROR:  deadlock detected
DETAIL:  Process 6633 waits for ShareLock on transaction 11960; 
 blocked by process 7735.
Process 7735 waits for ShareLock on transaction 11961; 
 blocked by process 6633.
HINT:  See server log for query details.

\end{verbatim}
\end{minipage}


Para responder às questões a seguir, execute o seguinte comando no
shell do linux:

\begin{verbatim}
$ curl --proxy-negotiate -u labfafram1 -s holanda.xyz/files/cidades.sql 2>&1 | psql
\end{verbatim}

\question{2} Qual o tipo de varredura e o custo das seguintes consultas:

\begin{enumerate}[a)]
\item {\tt SELECT nome FROM estado WHERE uf = 'AM';}
\item {\tt SELECT C.* FROM cidade AS C, estado AS E WHERE C.estado\_id=E.id AND E.nome= 'Acre';}
\end{enumerate}

\question{2} Qual a expressão SQL para criar índice para o campo {\tt
  uf} da tabela {\tt estado}? Escreva a expressão SQL que verifica o
espaço em disco ocupado pelo índice criado. Qual foi este espaço em
MB?

\question{1} Descreva se houve e quais foram as alterações para o
resultado da questão~1, com a adição do índice.

\question{1} Escreva a expressão SQL que cria uma visão para listar todas as 
cidades que pertencem ao estado de Goiás e do Distrito Federal.

\question{1} Escreva a expressão SQL que declara um cursor para a
listagem de todas as cidades do estado do Rio de Janeiro. Escreva a
expressão SQL que percorra 20 cidades por vez usando o cursor criado.

\question{1} Escreva a expressão SQL que modifica o {\tt id} da cidade
de São Paulo dentro de um transação.

\question{2} Escreva as expressões SQL para a criação de um esquema
chamado {\tt teste} que deverá conter a tabela {\tt estado}, sem o
campo {\tt pais\_id}.


